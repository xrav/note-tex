\Introduction

В настоящее время роботы вошли в жизнь человека в разных областях, но до сих пор нет четкого разделения среди робототехнических устройств, а также единой программной платформы. Разные производители делают разные и абсолютно несовместимые аппаратные средства. Технологии и инструменты затачиваются каждый раз под конкретные проблемы, их практически невозможно повторно использовать. Основной камень преткновения при создании роботов — искусственный интеллект. 

Какие проблемы стоят перед современной робототехникой?

\begin{itemize}
\item надёжная и доступная механика;
\item емкие и компактные элементы питания;
\item мощная и малопотребляющая вычислительная система;
\item точные и доступные датчики;
\item интеллектуальная система управления.
\end{itemize}

В рамках данной дипломной работы мы попытаемся решить задачу, которая лежит на стыке 3ех проблем: механики, вычислительной системы и системы управления.
Мы сформируем у робота мелкую моторику, которая будет заключаться в том, что мы научим робота писать простые символы. Другими словами, сформируем основы графомоторных навыков у робота на примере решения игры в Судоку.

Мы будем использовать робота LEGO MINDSTORMS [NXT] 2, который достаточно распространен у любителей робототехники.

Проблемы, которые придется нам решить в ходе данной дипломной работы это малое количество flash-памяти на роботе (256Кб) и недостаточная разрешающая способность блока с камерой (CAM-NXT).

Последовательность действий для  получения решения:
\begin{enumerate}
\item сфотографировать изображение;
\item распознать изображение;
\item сформировать решение.
\end{enumerate}
Для выполнения шага 1 и 2 нам не хватит встроенной памяти робота, так он не в состоянии даже сохранить фотографию в нужном нам качестве, поэтому решение данной задачи мы перенесем на внешнее мобильное устройство под управлением операционной системы Android. На телефоне мы распознаем изображение и отправим роботу посредством канала Bluetooth.

На выходе мы получаем:
\begin{enumerate}
\item простейшую систему распознавания цифр на телефоне, под управлением ОС Android;
\item протокол обмена между роботом и мобильным устройством;
\item система решения Судоку на роботе;
\item система отрисовки на бумаге полученного решения.
\end{enumerate}