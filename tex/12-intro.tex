\Introduction

В настоящее время роботы вошли в жизнь человека в разных областях, но до сих пор нет четкого разделения среди робототехнических устройств, а также единой программной платформы. Разные производители делают разные и абсолютно несовместимые аппаратные средства. Технологии и инструменты затачиваются каждый раз под конкретные проблемы, их практически невозможно повторно использовать. Основной камень преткновения при создании роботов — искусственный интеллект. 

Какие проблемы стоят перед современной робототехникой?

\begin{itemize}
\item надёжная и доступная механика;
\item емкие и компактные элементы питания;
\item мощная и малопотребляющая вычислительная система;
\item точные и доступные датчики;
\item интеллектуальная система управления.
\end{itemize}

В рамках данной дипломной работы мы попытаемся решить задачу, которая лежит на стыке 3ех проблем: механики, вычислительной системы и системы управления.
Мы сформируем у робота мелкую моторику, которая будет заключаться в том, что мы научим робота писать простые символы. Другими словами, сформируем основы графомоторных навыков у робота на примере решения игры в Судоку.

Целью дипломной работы является создание такого программного комплекса. В его задачи должны входить:
\begin{enumerate}
\item сфотографировать изображение;
\item распознать изображение;
\item сформировать решение,
\item вывод решения.
\end{enumerate}

После решения всех этих задач, мы сможем получить программно-аппаратный комплекс для демонстрации.

