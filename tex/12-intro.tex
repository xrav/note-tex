\Introduction

Целью работы является создание всякой всячины. Для достижения поставленной цели необходимо решить следующие задачи:

В настоящее время роботы вошли в жизнь человека в разных областях, но до сих пор нет четкого разделения среди робототехнических устройств, а также единой программной платформы. Разные производители делают разные и абсолютно несовместимые аппаратные средства. Технологии и инструменты затачиваются каждый раз под конкретные проблемы, их практически невозможно повторно использовать. Основной камень преткновения при создании роботов — искусственный интеллект. 

Какие проблемы стоят перед современной робототехникой?

\begin{itemize}
\item надёжная и доступная механика;
\item емкие и компактные элементы питания;
\item мощная и малопотребляющая вычислительная система;
\item точные и доступные датчики;
\item интеллектуальная система управления.
\end{itemize}

В рамках данной дипломной работы мы попытаемся решить задачу, которая лежит на стыке 3ех проблем: механики, вычислительной системы и системы управления.
Мы сформируем у робота мелкую моторику, которая будет заключаться в том, что мы научим робота писать простые символы . Другими словами, сформируем основы графомоторных навыков у робота на примере решения игры в Судоку.
