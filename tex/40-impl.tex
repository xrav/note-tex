\chapter{Технологический раздел}
\label{cha:impl}

В данном разделе описываются используемые при работе над проектом инструменты и технологии.

\section{Выбор языка программирования}

Для реализации системы выбран язык Java. Перечислим основные преимущества, говорящие в пользу его использования.
Java является объектно-ориентированным языком. Это дает возможность при разработке приложений использовать технологию объектно-ориентированного программирования, которая позволяет сократить общее время разработки и писать повторно используемый код.

Java-приложения являются независимыми от платформы. Это достигается путем совмещения в языке свойств компилятора и интерпретатора следующим образом: классы программы компилируются во внутренний байт-код, который может интерпретирован виртуальной java-машиной (JVM, Java Virtual Machine). 

Платформонезависимость байт-кода обеспечивается наличием виртуальных java-машин для всех основных платформ.

В комплект поставки Java (JDK, Java Developer Kit) входят стандартные классы, которые обладают достаточной функциональностью для быстрой разработки приложений.

Важным преимуществом является наличие большого количества библиотек, расширяющих возможности языка.

\section{Используемые библиотеки и модули}
\subsection{Мобильное приложение}

\paragraph{Android} — операционная система для смартфонов, планшетных компьютеров, электронных книг, цифровых проигрывателей, наручных часов, игровых приставок, нетбуков, смартбуков, других устройств. Основана на ядре Linux и собственной реализации Java от Google. 

Android позволяет создавать Java-приложения, управляющие устройством через разработанные Google библиотеки. Android Native Development Kit позволяет портировать библиотеки и компоненты приложений.

\subsection{Android SDK}

Приложения под операционную систему Android являются программами в нестандартном байт-коде для виртуальной машины Dalvik, для них был разработан формат установочных пакетов .APK.

Google предлагает для свободного скачивания инструментарий для разработки (Software Development Kit), который предназначен для x86-машин под операционными системами Linux, Mac OS X (10.4.8 или выше), Windows XP, Windows Vista и Windows 7. Для разработки требуется JDK 5 или более новый.

Разработку приложений для Android можно вести на языке Java (не ниже Java 1.5).

Существует плагин для Eclipse — Android Development Tools (ADT), предназначенный для Eclipse версий 3.3—3.7. Также существует плагин для IntelliJ IDEA, облегчающий разработку Android-приложений, и для среды разработки NetBeans IDE, который, начиная с версии NetBeans 7.0, перестал быть экспериментальным, хоть пока и не является официальным.

Кроме того, существует Motodev Studio for Android — комплексная среда разработки на базе Eclipse, позволяющая работать непосредственно с Google SDK.

В целом, \textbf{Andoroid SDK} предоставляет нам все необходимые средства для реализации заявленных задач.

\subsection{ПО для контроллера робота}

\paragraph{leJOS} является заменой прошивки для программируемых блоков Lego Mindstorms.

В настоящее время она поддерживает LEGO RCX блок и Lejos NXJ поддерживает блок NXT. Она включает в себя виртуальную машину Java, что позволяет Lego Mindstorms роботам писать программное обеспечение на языке программирования Java. 

Так как Lejos NXJ является проектом Java, она основывается на богатстве функциональных возможностей, присущей платформе Java. Существуют Lejos NXJ плагины для двух ведущих IDE для Java: Eclipse и NetBeans. Разработчики могут воспользоваться стандартной функциональности в IDE (автозавершения кода, рефакторинга и шаблонов для тестирования), а также такие возможности как: компиляция, сборка и выгрузки. 

Lejos NXJ обеспечивает поддержку доступа к портам робота. Это позволяет получить доступ к стандартным датчикам и двигателям (ультразвуковой датчик расстояния, сенсорный датчик, звук и датчик света). Другие компании, такие как MindSensors и HiTechnic распространили этот базовый набор, предоставляя усовершенствованные датчики, исполнительные механизмы и мультиплексоры. Lejos NXJ включает Java API, для этих продуктов.

Воспользовавшись объектно-ориентированной структурой языка Java, разработчики Lejos NXJ смогли скрыть детали реализации датчиков. Это позволяет разработчику работать с абстракциями высокого уровня, не беспокоясь о деталях, например таких как шестнадцатеричная адресация аппаратных компонентов.

Проект включает в себя реализацию наиболее часто используемых контроллером обратной связи, ПИД-регулятора и алгоритма снижения фильтр шума Калмана. Lejos NXJ также предоставляет библиотеки, которые поддерживают более абстрактные функции, такие как навигации, картографирования и поведения.

\begin{lstlisting}[caption={Пример кода для работы с двигателями}, language=Java]
import lejos.nxt.Motor;
import lejos.nxt.Button;

public class Example {
    public static void main(String[] args) {
        Motor.A.forward();
        Button.waitForPress();
        Motor.A.backward();
        Button.waitForPress();
        System.exit(1);
    }
}
\end{lstlisting}


Таким образом, мы можем писать на удобном нам языке (в данном случае Java) и работать с подключаемыми устройствами робота посредством API, просто делая импорт нужной библиотеки, которая отвечает за управляющие команды или данные сенсоров.

\section{Система контроля версий}
\subsection{Git}

В качестве системы контроля версий была выбрана CVS Git. Данный инструмент является очень популярным и удобным в работе. Не требует установки сервера при одиночной разработке.

Работа с подмодулями в данной системе упрощает разработку и интеграцию нескольких проектов параллельно, что является большим плюсом в работе, поскольку каждый компонент системы был оформлен как отдельный репозиторий git.


\section{Среда разработки}
Выбор описанных выше инструментов привел к естественному выбору IDE, поддерживающей как язык Java и работу с его зависимостями и библиотеками, так и обладающей интеграцией с CVS Git. Такой средой стала программа IntelliJ IDEA компании JetBrains. 
\subsection{IntelliJ IDEA}

Данная среда разработки создана специально для разработки на языке Java и имеет отличные инструменты для отладки программ, запуска модульных и функциональных тестов с возможностью отображения покрытия кода тестами, эмулятор командой строки для быстрого запуска консольных команд, систему управления библиотеками Java, а так же поддержка всевозможных Java фреймворков и библиотек и всех его основных функций. 

Подсветка синтаксиса, автодополнение, инспекция кода и другие функции облегающие жизнь разработчика так же присутствуют в данной IDE, а прекрасное выполнения функции поиска классов и файлов проекта экономит значительное количество времени.

Так же кроме написания ПО на языке Java, IntelliJ IDEA поддерживает работу с другими языками, документами HTML, CSS-стилями и JavaScript-скриптами. Поддержка манифест-файлов для разного рода приложений, умение понимать, анлизировать и определять вставки исходного кода на другом языке облегчают процесс разработки.

Таким образом данная программа является идеальным решением для разработки на языке Java и в частности Java, Android и других производных приложений.

\section{База данных}
\subsection{SQLite}
\paragraph{SQLite} - компактная встраиваемая реляционная база данных. Исходный код библиотеки передан в общественное достояние.

Слово «встраиваемый» означает, что SQLite не использует парадигму клиент-сервер, то есть СУБД SQLite не является отдельно работающим процессом, с которым взаимодействует программа, а предоставляет библиотеку, с которой программа компонуется и СУБД становится составной частью программы. Таким образом, в качестве протокола обмена используются вызовы функций (API) библиотеки SQLite. Такой подход уменьшает накладные расходы, время отклика и упрощает программу.

SQLite хранит всю базу данных (включая определения, таблицы, индексы и данные) в единственном стандартном файле на том компьютере, на котором исполняется программа. Простота реализации достигается за счёт того, что перед началом исполнения транзакции записи весь файл, хранящий базу данных, блокируется; ACID-функции достигаются в том числе за счёт создания файла журнала.

Несколько процессов или потоков могут одновременно без каких-либо проблем читать данные из одной базы. Запись в базу можно осуществить только в том случае, если никаких других запросов в данный момент не обслуживается; в противном случае попытка записи оканчивается неудачей, и в программу возвращается код ошибки. Другим вариантом развития событий является автоматическое повторение попыток записи в течение заданного интервала времени.

В комплекте поставки идёт также функциональная клиентская часть в виде исполняемого файла sqlite3, с помощью которого демонстрируется реализация функций основной библиотеки. Клиентская часть работает из командной строки, позволяет обращаться к файлу БД на основе типовых функций ОС.

Благодаря архитектуре СУБД возможно использовать SQLite как на встраиваемых системах, так и на выделенных машинах с гигабайтными массивами данных.

Таким образом SQLite идеально подходит для реляционного хранения данных на мобильном устройстве под управлением ОС Android.


%%% Local Variables:
%%% mode: latex
%%% TeX-master: "rpz"
%%% End:
