\Conclusion % заключение к отчёту

В процессе работы были решены следующие задачи:

\begin{enumerate}
    \item проведена классификация и анализ мобильных роботов;
	\item разработана система распознавания изображения;
	\item реализован протокол обмена между роботом и мобильным устройством; 
	\item реализована система решения Судоку;
	\item разработать систему вывода полученного решения манипулятором робота;
	\item исследована скорость работы алгоритмов.
\end{enumerate}

Разработан программно-аппаратный комплекс, включающий сопряженных робота и мобильное устройство. Программный комплекс позволяет распознавать на мобильном устройстве задание для Судоку, на роботе решать его и выводить решение на мобильное устройство и при помощи манипулятора робота ручкой на бумагу.

Реализованы графомоторные навыки робота: робот определяет область вывода и выводит символы из заложенного в него алфавита.

Предоставлена возможность для разработчиков перейти на другой уровень абстракций, уйдя от машинных команд. Реализованный интерфейс позволяет подавать роботу команды вывода символов в пределах заложенного в робот алфавита.

Существуют следующие направления по улучшению программного комплекса:

\begin{enumerate}
    \item в процессе распознавания операции выполняются попиксельно, процесс можно распараллелить за счет подключения плат, которые предоставляют эту возможность;
	\item уменьшить время вывода символов за счет использования альтернативных моторов;
	\item с увеличением мощностей контроллеров и увеличением памяти добиться автономности (все подсистемы будут на роботе).
\end{enumerate}


Также был проведен анализ экономической эффективности разработанной системы и приведены оптимальные показатели условий труда при работе над продуктом.



%%% Local Variables: 
%%% mode: latex
%%% TeX-master: "rpz"
%%% End: 
