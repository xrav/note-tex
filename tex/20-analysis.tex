\chapter{Аналитический раздел}
\label{cha:analysis}
%
% % В начале раздела  можно напомнить его цель
%
В данном разделе производится анализ процессов распознавания изображения, передачи распознанного изображения, приема изображения и  решения решение его на роботе.
Производится анализ подсистем, входящих в реализуемы программно-аппаратный комплекс, формируются требования к создаваемой системе, выделяются функции её подсистем и описывается взаимодействие между ними.

\section{Общее описание системы}
В рамках данной дипломной работы разрабатывается автономный программно-аппаратный комплекс предоставляющий полный цикл для передачи и распознавания информации, для функционирования которой необходимо спроектировать все подсистемы программного комплекса и способы их взаимодействия.

Для достижения поставленной цели необходимо спроектировать и разработать следующие компоненты системы:
\begin{enumerate}
\item разработать систему распознавания изображения;
\item разработать протокол обмена между роботом и мобильным устройством;
\item разработать систему решения Судоку;
\item разработать систему отрисовки на бумаге полученного решения.
\end{enumerate}

\begin{figure}
  \centering
  \includegraphics[width=\textwidth]{inc/dia/analysis1-1}
  \caption{Схема работы комплекса}
  \label{fig:fig01}
\end{figure}


Так же для удобства и простоты использования комплекса необходимо разработать мобильное приложение и протокол обмена между мобильным устройством и роботом, для того, чтобы стандартизировать сообщения и разрабатывать приложения для любых внешних платформ.

\begin{figure}
  \centering
  \includegraphics[width=\textwidth]{inc/dia/analysis1-2}
  \caption{Компоненты системы}
  \label{fig:fig02}
\end{figure}

\section{Классификация мобильных роботов}
\paragraph{Мобильный робот} - автоматическая машина, в которой имеется движущееся шасси с автоматически управляемыми приводами. Такие роботы могут быть колёсными, шагающими и гусеничными (существуют также ползающие, плавающие и летающие мобильные робототехнические системы).

\paragraph{Микроконтроллер, микрокомпьютеры (англ. Micro Controller Unit, MCU)} — микросхема, предназначенная для управления электронными устройствами. Типичный микроконтроллер сочетает на одном кристалле функции процессора и периферийных устройств, содержит ОЗУ и (или) ПЗУ. 

Поскольку основная масса роботов отличается микроконтроллерами, введем классификацию по этому управляющему элементу.

Существующие робототехнические комплексы для учебных лабораторий:
\begin{enumerate}
\item LEGO Mindstorms;
\item fischertechnik.
\end{enumerate}

\subsection{LEGO Mindstorms}

\paragraph{LEGO Mindstorms} - конструктор (набор сопрягаемых деталей и электронных блоков) для создания программируемого робота. 

Наборы LEGO Mindstorms комплектуются набором стандартных деталей LEGO (балки, оси, колеса, шестерни) и набором, состоящим из сенсоров, двигателей и программируемого блока. Наборы делятся на базовый и ресурсный.

В состав наборов могут входить управляющие блоки различных версий. В настоящее время их 3. Также у блоков существуют модификации (обозначается 1.0; 2.0 и т. д.)
\begin{enumerate}
\item RCX;
\item NXT;
\item EV3.
\end{enumerate}
Наборы LEGO Mindstorms располагают огромным количеством сенсоров как компании LEGO, так и сторонних производителей (HiTechnic, Mindsensors). 


\paragraph{Мобильный робот}

\section{Цель и задачи}
Целью данной работы является решение головоломки Судоку роботом и вывод решения роботом на лист бумаги.


Входные данные.
\begin{itemize}
\item головоломка судоку на листе бумаги.
\end{itemize}

Выходные данные.
\begin{itemize}
\item решение головоломки на листе бумаги.
\end{itemize}

Последовательность действий для  получения решения:
\begin{enumerate}
\item сфотографировать изображение;
\item распознать изображение;
\item сформировать решение.
\end{enumerate}
Для выполнения шага 1 и 2 нам не хватит встроенной памяти робота, так он не в состоянии даже сохранить фотографию в нужном нам качестве, поэтому решение данной задачи мы перенесем на внешнее мобильное устройство под управлением операционной системы Android. На телефоне мы распознаем изображение и отправим роботу посредством канала Bluetooth.

На выходе мы получаем:
\begin{enumerate}
\item простейшую систему распознавания цифр на телефоне, под управлением ОС Android;
\item протокол обмена между роботом и мобильным устройством;
\item система решения Судоку на роботе;
\item система отрисовки на бумаге полученного решения.
\end{enumerate}

% Обратите внимание, что включается не ../dia/..., а inc/dia/...
% В Makefile есть соответствующее правило для inc/dia/*.pdf, которое
% берет исходные файлы из ../dia в этом случае.




\begin{figure}[ht!]
 \centering 
 \includegraphics[width=\textwidth]{inc/raster/sample.jpg} 
 \caption{A simple caption} 
 \label{overflow} 
\end{figure}



\section{Какой-то последний раздел}
Проблемы, которые придется нам решить в ходе данной дипломной работы это малое количество flash-памяти на роботе (256Кб) и недостаточная разрешающая способность блока с камерой (CAM-NXT).

%%% Local Variables:
%%% mode: latex
%%% TeX-master: "rpz"
%%% End:
