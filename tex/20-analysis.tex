\chapter{Аналитический раздел}
\label{cha:analysis}
%
% % В начале раздела  можно напомнить его цель
%
В данном разделе производится анализ процессов распознавания изображения, передачи распознанного изображения, приема изображения и  решения решение его на роботе.
Производится анализ подсистем, входящих в реализуемы программный комплекс, формируются требования к создаваемой системе, выделяются функции её подсистем и описывается взаимодействие между ними.

\section{Цель и задачи}
Целью данной работы является решение головоломки Судоку роботом и вывод решения роботом на лист бумаги.

Для достижения данной цели необходимо решить следующие задачи:
\begin{enumerate}
\item разработать систему распознавания изображения;
\item ???разработать протокол обмена между роботом и мобильным устройством;
\item разработать систему решения Судоку;
\item разработать систему отрисовки на бумаге полученного решения.
\end{enumerate}

Входные данные.
\begin{itemize}
\item головоломка судоку на листе бумаги.
\end{itemize}

Выходные данные.
\begin{itemize}
\item решение головоломки на листе бумаги.
\end{itemize}

Последовательность действий для  получения решения:
\begin{enumerate}
\item сфотографировать изображение;
\item распознать изображение;
\item сформировать решение.
\end{enumerate}
Для выполнения шага 1 и 2 нам не хватит встроенной памяти робота, так он не в состоянии даже сохранить фотографию в нужном нам качестве, поэтому решение данной задачи мы перенесем на внешнее мобильное устройство под управлением операционной системы Android. На телефоне мы распознаем изображение и отправим роботу посредством канала Bluetooth.

На выходе мы получаем:
\begin{enumerate}
\item простейшую систему распознавания цифр на телефоне, под управлением ОС Android;
\item протокол обмена между роботом и мобильным устройством;
\item система решения Судоку на роботе;
\item система отрисовки на бумаге полученного решения.
\end{enumerate}

% Обратите внимание, что включается не ../dia/..., а inc/dia/...
% В Makefile есть соответствующее правило для inc/dia/*.pdf, которое
% берет исходные файлы из ../dia в этом случае.

\section{Подсистема предоставления доступа к данных}

\subsection{Общие представления о системе}

Главной задачей разрабатываемой системы является предоставление доступа сторонним приложениям доступ к массивам данных, хранящихся на серверах системы. Как известно, "чтобы купить что-нибудь ненужное, нужно сначала купить что-нибудь ненужное", и эту задачу призвана решить первая рассматриваемая подсистема - подсистема сбора данных. Второй важнейшей задачей является сохранение данных, а так же и непосредственная передача клиенту. Эту проблему призвано решить вторая рассматриваемая подсистема.

Очевидно, что в качестве сердца данной подсистемы выступает некоторая база данных, которую необходимо наполнить данными, а затем и предоставить доступ к этим данным. Для решения это задачи должно быть разработано приложения, предоставляющий некоторый интерфейс для доступа к базе данных, так как прямой доступ к базе крайне нежелателен, а в некоторых случаях даже опасен. Такой интерфейс и будет предоставлять второе приложение.

\begin{figure}[ht!]
 \centering 
 \includegraphics[width=\textwidth]{inc/raster/sample.jpg} 
 \caption{A simple caption} 
 \label{overflow} 
\end{figure}

Мы будем использовать робота LEGO MINDSTORMS [NXT] 2, который достаточно распространен у любителей робототехники.
\begin{figure}
  \centering
  \includegraphics[width=\textwidth]{inc/dia/analysis1-1}
  \caption{Схема работы подсистемы управления данными}
  \label{fig:fig05}
\end{figure}

\section{Какой-то последний раздел}
Проблемы, которые придется нам решить в ходе данной дипломной работы это малое количество flash-памяти на роботе (256Кб) и недостаточная разрешающая способность блока с камерой (CAM-NXT).

%%% Local Variables:
%%% mode: latex
%%% TeX-master: "rpz"
%%% End:
