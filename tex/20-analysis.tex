\chapter{Аналитический раздел}
\label{cha:analysis}
%
% % В начале раздела  можно напомнить его цель
%
В данном разделе производится анализ процессов распознавания изображения, передачи распознанного изображения, приема изображения и  решения решение его на роботе.
Производится анализ подсистем, входящих в реализуемы программно-аппаратный комплекс, формируются требования к создаваемой системе, выделяются функции её подсистем и описывается взаимодействие между ними.

\section{Общее описание системы}
В рамках данной дипломной работы разрабатывается автономный программно-аппаратный комплекс предоставляющий полный цикл для передачи и распознавания информации, для функционирования которой необходимо спроектировать все подсистемы программного комплекса и способы их взаимодействия.

Для достижения поставленной цели необходимо спроектировать и разработать следующие компоненты системы:
\begin{enumerate}
\item разработать систему распознавания изображения;
\item разработать протокол обмена между роботом и мобильным устройством;
\item разработать систему решения Судоку;
\item разработать систему отрисовки на бумаге полученного решения.
\end{enumerate}


% Обратите внимание, что включается не ../dia/..., а inc/dia/...
% В Makefile есть соответствующее правило для inc/dia/*.pdf, которое
% берет исходные файлы из ../dia в этом случае.


\begin{figure}
  \centering
  \includegraphics[width=\textwidth]{inc/dia/analysis1-1}
  \caption{Схема работы комплекса}
  \label{fig:fig01}
\end{figure}


Так же для удобства и простоты использования комплекса необходимо разработать мобильное приложение и протокол обмена между мобильным устройством и роботом, для того, чтобы стандартизировать сообщения и разрабатывать приложения для любых внешних платформ.

\begin{figure}
  \centering
  \includegraphics[width=\textwidth]{inc/dia/analysis1-2}
  \caption{Компоненты системы}
  \label{fig:fig02}
\end{figure}

\section{Классификация мобильных роботов}
\paragraph{Мобильный робот} - автоматическая машина, в которой имеется движущееся шасси с автоматически управляемыми приводами. Такие роботы могут быть колёсными, шагающими и гусеничными (существуют также ползающие, плавающие и летающие мобильные робототехнические системы).

\paragraph{Микроконтроллер, микрокомпьютеры (англ. Micro Controller Unit, MCU)} — микросхема, предназначенная для управления электронными устройствами. Типичный микроконтроллер сочетает на одном кристалле функции процессора и периферийных устройств, содержит ОЗУ и (или) ПЗУ. 

Поскольку основная масса роботов отличается микроконтроллерами, введем классификацию по этому управляющему элементу.

Существующие робототехнические комплексы:
\begin{itemize}
\item LEGO Mindstorms;
\item fischertechnik.
\end{itemize}

\subsection{LEGO Mindstorms}

\paragraph{LEGO Mindstorms} - конструктор (набор сопрягаемых деталей и электронных блоков) для создания программируемого робота. 

Наборы LEGO Mindstorms комплектуются набором стандартных деталей LEGO (балки, оси, колеса, шестерни) и набором, состоящим из сенсоров, двигателей и программируемого блока. Наборы делятся на базовый и ресурсный.

Все наборы содержат в себе одну и ту же версию интеллектуального блока NXT, отличаются только версии прошивки, но это не принципиально, так как прошивку можно легко обновить. Так что в этом плане все наборы совершенно равноценны. 

В состав наборов могут входить управляющие блоки различных версий. В настоящее время их 3. Также у блоков существуют модификации (обозначается 1.0; 2.0 и т. д.)

Управляющие блоки:
\begin{itemize}
\item RCX - первое поколение управляющих блоков, в данный момент почти не используются из-за устаревшей конструкции модели;
\item NXT - вторая версия коммерческого набора и самое распространенное поколение, 619 деталей в базовом комплекте, год выпуска 2009;
\item EV3 - третье поколение, эволюция модели NXT, более 550 деталей, представлен в сентябре 2013 года.
\end{itemize}
Наборы LEGO Mindstorms располагают огромным количеством сенсоров как компании LEGO, так и сторонних производителей (HiTechnic, Mindsensors и др.). 

\subsection{fischertechnik}

\paragraph{fischertechnik} — пластмассовый развивающий конструктор для детей, подростков и студентов, изобретенный профессором Артуром Фишером в 1964 году. Наборы \textbf{fischertechnik} выпускает немецкая фирма \textbf{fischertechnik GmbH}, которая входит в состав крупного холдинга \textbf{fischertechnik GmbH \& Co.KG}, дочерние фирмы которого выпускают крепёж, крепежный инструмент, детали для автомобилей и различные изделия из пластмассы.

Конструкторы fischertechnik часто используются для демонстрации принципов работы механизмов и машин в средних, специальных и высших учебных заведениях, а также для моделирования производственных процессов и презентационных целей.

Также в комплекты конструкторов входят программируемые контроллеры, двигатели, различные датчики и блоки питания, что позволяет приводить механические конструкции в движение, создавать роботов и программировать их с помощью компьютера.

Имеют только один вид контроллера: \textbf{ROBO TX} - это компактный программируемый контроллер для управления моделями, собранными из конструкторов fischertechnik.

Для разработки управляющих программ для контроллера ROBO TX используется среда программирования \textbf{ROBO Pro}. Готовые программы загружаются в контроллер через интерфейсы USB или Bluetooth.

\subsection{Прочие робототехнические комплексы}

Прочие робототехнические комплексы не представляют никакого интереса для изучения, т.к.  имеют контроллеры специфичной конфигурации, узкое количество сенсоров и наборов, плохо документированные и поддерживаемые узкой группой энтузиастов среды для программирования.

\subsection{Вывод}
Мы будем использовать для написания данного дипломного проекта робототехнический комплекс \textbf{LEGO Mindstorms} по нескольким причинам:
\begin{itemize}
\item большое количество подключаемых модулей, как от самой компании производителя, так и от сторонних компаний;
\item отличная документация на разных языках;
\item самое большое сообщество робототехников, которые поддерживают и развивают данный комплекс;
\item использование альтернативного ПО для программирования роботов.
\end{itemize}

В базовый комплект поставки Mindstorms NXT 2.0 уже включено большинство необходимых деталей для выполнения практически любых задач.


\section{Подсистема распознавания}
\subsection{Общие представления о системе}

Подсистема распознавания данных должна решать проблему перевода изображений и формирования из него структурированных данных, для последующей передачи другим подсистемам. Таким образом, распознанное изображение должно принять удобный для хранения, использования и передачи, структурированный вид. Результатом работы такой системы должны быть данные, готовые к автоматизированной обработке. 

Общая схема процесса представлена на (рис.~\ref{fig:fig03}).
\begin{figure}
  \centering
  \includegraphics[width=\textwidth]{inc/dia/analysis1-3}
  \caption{Подсистема сбора данных}
  \label{fig:fig03}
\end{figure}


\subsection{Существующие системы распознавания изображений}

На данный момент существует огромное количество программ, поддерживающих распознавание текста как одну из возможностей. Мы не будем рассматривать такие системы, так как в большинстве своем они избыточны, а наш программно-аппаратный комплекс будет работать в условиях ограниченных аппаратных ресурсов.

\paragraph{OpenCV} (англ. Open Source Computer Vision Library, библиотека компьютерного зрения с открытым исходным кодом) — библиотека алгоритмов компьютерного зрения, обработки изображений и численных алгоритмов общего назначения с открытым кодом. Реализована на C/C++, также разрабатывается для Python, Java, Ruby, Matlab, Lua и других языков. Может свободно использоваться в академических и коммерческих целях — распространяется в условиях лицензии BSD.

\paragraph{JavaANPR}. Реализация – Java. Проект располагается по адресу \url{http://javaanpr.sourceforge.net}. Основное преимущество этой библиотеки в ее кроссплатформенности. Кроме этого, все алгоритмы написаны на Java без использования нативных библиотек, что сильно упрощает использование. Так же эту библиотеку с небольшой доработкой можно использовать на устройствах под управлением OS Android.

Огромный список подобного рода систем показывает, что распознавание информации является популярной и сложной задачей. Для универсального решения данной задачи требуется использовать сложный программно-аппаратный комплекс, требующий огромных вычислительных мощностей(для наших мобильных устройств), что выходит за рамки данной работы и может воплотиться в виде развития рассматриваемой темы. 

\subsection{Возможности системы}

Подсистема распознавания сбора данных должна обеспечивать получение изображение с камеры устройства, распознавание его и сохранение в нужной структуре.  Таким образом приложение, реализующее данную систему, должно работать на мобильном устройстве и должна сохранять изображение в формате, пригодном для дальнейшего решения. 

Для получения чисел из сетки судоку, нам надо определить, а где же наша сетка начинается и кончается. Эта часть является простейшей частью для человеческого мозга, но самой сложной для ПО. Почему? В них слишком много лишних данных. Очень часто газеты и журналы печатают несколько судоку рядом (хм, они явно не рассчитывают на компьютерное распознавание последних). На изображении будет слишком много лишних линий. 

ПО очень сложно определить, какие линии относятся к необходимым нам, а какие являются всего лишь информационным шумом. Где конец нашей сетки и начало следующей.

Каждый алгоритм распознавания имеет три шага:
\begin{itemize}
\item определение необходимых признаков,
\item тренировка,
\item классификация (распознавание в реальном времени).
\end{itemize}

То есть на выходе мы должны иметь упорядоченную структуру в алфавите от 1 до 9.

\subsection{Процесс распознавания}

Для получения чисел из сетки судоку, нам надо определить, а где же наша сетка начинается и кончается.  На изображении будет слишком много лишних линий. 

Компьютеру очень сложно определить, какие линии относятся к необходимым нам, а какие являются всего лишь информационным шумом. Где конец нашей сетки и начало следующей.

После того, как мы определили границы, запустим алгоритм преобразования чтобы точно определить линии сетки. До сих пор мы не заботились о перекосах и других дефектах изображения. Только об угле поворота. Этот шаг исправит это. Мы получим точные положения линий сетки. Это поможет определить числа в сетке.

После того, как мы определили, где должны находиться числа, нам необходимо распознать их. Это относительно легко. В алфавите только цифры от 1 до 9.

Данный процесс представлен на (рис.~\ref{fig:fig04}).
\begin{figure}
  \centering
  \includegraphics[width=\textwidth]{inc/dia/analysis1-4}
  \caption{Схема работы подсистемы сбора данных}
  \label{fig:fig04}
\end{figure}


\subsection{Стадии обработки изображения}

\subsection{Требования к подсистеме}

\subsection{Требования к реализации}

\subsection{Требования к надежности}

\subsection{Входные данные подсистемы}

\begin{enumerate}
 \item Изображение, полученное с камеры мобильного устройства.
\end{enumerate}

\subsection{Выходные параметры}
\begin{enumerate}
 \item Результат обработки заявки в виде некоторой структуры данных;
 \item в случае, если распознавание не прошло полный цикл обработки, данные отправляются для повторного распознавания;
 \item информация о состоянии распознавания.
\end{enumerate}

%%% Local Variables:
%%% mode: latex
%%% TeX-master: "rpz"
%%% End:
