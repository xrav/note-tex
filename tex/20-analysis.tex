\chapter{Аналитический раздел}
\label{cha:analysis}
%
% % В начале раздела  можно напомнить его цель
%
В данном разделе производится анализ процессов распознавания изображения, передачи распознанного изображения, приема изображения и  решения решение его на роботе.
Производится анализ подсистем, входящих в реализуемы программный комплекс, формируются требования к создаваемой системе, выделяются функции её подсистем и описывается взаимодействие между ними.

% Обратите внимание, что включается не ../dia/..., а inc/dia/...
% В Makefile есть соответствующее правило для inc/dia/*.pdf, которое
% берет исходные файлы из ../dia в этом случае.

\section{Подсистема предоставления доступа к данных}

\subsection{Общие представления о системе}

Главной задачей разрабатываемой системы является предоставление доступа сторонним приложениям доступ к массивам данных, хранящихся на серверах системы. Как известно, "чтобы купить что-нибудь ненужное, нужно сначала купить что-нибудь ненужное", и эту задачу призвана решить первая рассматриваемая подсистема - подсистема сбора данных. Второй важнейшей задачей является сохранение данных, а так же и непосредственная передача клиенту. Эту проблему призвано решить вторая рассматриваемая подсистема.

Очевидно, что в качестве сердца данной подсистемы выступает некоторая база данных, которую необходимо наполнить данными, а затем и предоставить доступ к этим данным. Для решения это задачи должно быть разработано приложения, предоставляющий некоторый интерфейс для доступа к базе данных, так как прямой доступ к базе крайне нежелателен, а в некоторых случаях даже опасен. Такой интерфейс и будет предоставлять второе приложение.

\begin{figure}
  \centering
  \includegraphics[width=\textwidth]{inc/dia/analysis1-5}
  \caption{Схема работы подсистемы управления данными}
  \label{fig:fig05}
\end{figure}


%%% Local Variables:
%%% mode: latex
%%% TeX-master: "rpz"
%%% End:
